\section{Ejercicio 2}
\subsection{Código}
\inputminted[fontsize=\footnotesize]{systemverilog}{../arith_operator/rtl/arith_operator.sv}

\subsection{Verificación}
Se realizaron 5 TCs, que se pueden encontrar en este \href{https://github.com/msebgarcia/DDA2024/blob/main/GP01/arith_operator/tb/test_module.py}{link de github}, utilizando cocotb al igual que en el inciso anterior:
\begin{itemize}
    \item TC001: Suma con datos de entrada aleatorios
        \begin{minted}[fontsize=\footnotesize]{python}
async def TC001(dut):
    await init_test(dut)
    dut.i_sel.value    = 0
    for _ in range(100):
        dut.i_data_a.value = rnd.randint(-2**(int(dut.NB_DATA.value)-1),2**(int(dut.NB_DATA.value)-1)-1)
        dut.i_data_b.value = rnd.randint(-2**(int(dut.NB_DATA.value)-1),2**(int(dut.NB_DATA.value)-1)-1)
        await Timer(rnd.randint(10,50), 'ns')
        \end{minted}

    \item TC002: Resta con datos de entrada aleatorios
        \begin{minted}[fontsize=\footnotesize]{python}
async def TC002(dut):
    await init_test(dut)
    dut.i_sel.value    = 1
    for _ in range(100):
        dut.i_data_a.value = rnd.randint(-2**(int(dut.NB_DATA.value)-1),2**(int(dut.NB_DATA.value)-1)-1)
        dut.i_data_b.value = rnd.randint(-2**(int(dut.NB_DATA.value)-1),2**(int(dut.NB_DATA.value)-1)-1)
        await Timer(rnd.randint(10,50), 'ns')
        \end{minted}

    \item TC003: Operación And  con datos de entrada aleatorios
        \begin{minted}[fontsize=\footnotesize]{python}
async def TC003(dut):
    await init_test(dut)
    dut.i_sel.value    = 2
    for _ in range(100):
        dut.i_data_a.value = rnd.randint(-2**(int(dut.NB_DATA.value)-1),2**(int(dut.NB_DATA.value)-1)-1)
        dut.i_data_b.value = rnd.randint(-2**(int(dut.NB_DATA.value)-1),2**(int(dut.NB_DATA.value)-1)-1)
        await Timer(rnd.randint(10,50), 'ns')
        \end{minted}

    \item TC004: Operación Or con datos de entrada aleatorios
        \begin{minted}[fontsize=\footnotesize]{python}
async def TC004(dut):
    await init_test(dut)
    dut.i_sel.value    = 3
    for _ in range(100):
        dut.i_data_a.value = rnd.randint(-2**(int(dut.NB_DATA.value)-1),2**(int(dut.NB_DATA.value)-1)-1)
        dut.i_data_b.value = rnd.randint(-2**(int(dut.NB_DATA.value)-1),2**(int(dut.NB_DATA.value)-1)-1)
        await Timer(rnd.randint(10,50), 'ns')
        \end{minted}

    \item TC005: Randomización de operación y datos de entrada
        \begin{minted}[fontsize=\footnotesize]{python}
async def TC005(dut):
    await init_test(dut)
    for _ in range(100):
        dut.i_sel.value    = rnd.randint(0, 2**int(dut.NB_SEL.value)-1)
        dut.i_data_a.value = rnd.randint(-2**(int(dut.NB_DATA.value)-1),2**(int(dut.NB_DATA.value)-1)-1)
        dut.i_data_b.value = rnd.randint(-2**(int(dut.NB_DATA.value)-1),2**(int(dut.NB_DATA.value)-1)-1)
        await Timer(rnd.randint(10,50), 'ns')
        \end{minted}
\end{itemize}

La función common para inicializar el test es:
\begin{minted}[fontsize=\footnotesize]{python}
async def init_test(dut):
    tester = RtlModel(dut)

    dut.i_sel.value    = 0
    dut.i_data_a.value = 0
    dut.i_data_b.value = 0
    await Timer(rnd.randint(10,50), 'ns')
    tester.start()
\end{minted}

La función para modelar el RTL es:
\begin{minted}[fontsize=\scriptsize]{python}
async def run_beh(self, dut) -> None:
    while (True):
        await Timer(1, 'ns')
        obtained_data = dut.o_data_c.value
        if   (self.dut.i_sel.value == LogicArray('00', Range(int(dut.NB_SEL.value)-1, 'downto', 0))):
            assert obtained_data == LogicArray.from_signed(int(dut.i_data_a.value) + int(dut.i_data_b.value), Range(int(self.dut.NB_DATA.value)+1, "downto", 0))[int(self.dut.NB_DATA.value)-1:0]
        elif (self.dut.i_sel.value == LogicArray('01', Range(int(dut.NB_SEL.value)-1, 'downto', 0))):
            assert obtained_data == LogicArray.from_signed(int(dut.i_data_a.value) - int(dut.i_data_b.value), Range(int(self.dut.NB_DATA.value)+1, "downto", 0))[int(self.dut.NB_DATA.value)-1:0]
        elif (self.dut.i_sel.value == LogicArray('10', Range(int(dut.NB_SEL.value)-1, 'downto', 0))):
            assert dut.o_data_c.value == dut.i_data_a.value & dut.i_data_b.value
        else:
            assert obtained_data == dut.i_data_a.value or dut.i_data_b.value
\end{minted}

En este caso, tanto el modelo del RTL como los TCs se colocaron en el mismo archivo por simplicidad. También se subió el .vcd donde se dumpearon las señales.

